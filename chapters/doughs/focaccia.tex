\setchapterstyle{kao}
\setchapterpreamble[u]{\margintoc}
\chapter{Frank-Focaccia-stein}

This is my original recipe of Focaccia, based on Chiara's super special focaccia and Luiz's super tasty one.

\section{Ingredients}

\begin{multicols}{2}
\begin{description}
	\item[150 g] Semola di grano duro rimacinata
	\item[150 g] Manitoba (wheat min.13g)
	\item[250 g] Wheat flour (tipo 0)
	
	\item[15 g] Yeast
	\item[11 g] Salt
	\item[200 g] Milk
	\item[30 g] Oil \gls{evo}
	
	
	\item[150 g] Water
	\item[$\infty$] \gls{evo} to slather on
\end{description}
\end{multicols}	

\section{Preparation}
Mix the water, wheat flour, and yeast.
%
Cover and let it rest for a $1$ hour, or in the fridge overnight \sidenote{You can put less yeast and increase the fermentation time, which will add more flavour. This is the so called Biga. Typical recipes call for $1\%$ yeast for an $16h$ fermentation or $24h$ in the fridge \href{https://www.dissapore.com/cucina/biga-cose-preparazione-e-usi/}{link}.}.

Mix all the ingredients, cover, and leave it to rest for $30$ minutes before kneading\sidenote{Read Chapter \ref{burger buns} for the whys in this recipe.}.
%
Relax and clean the space in the meanwhile.

After kneading, shape it into a tight ball and leave it to rise for $1h$.
%
Then, reshape it into a ball carefully, you do not need to completely deflate the dough.
%
After that, rest in a bow with oil for another $30$ minutes\sidenote{that gurl needs to relax a bit before being opened.}.

\marginnote[1.5cm]{
	\begin{kaobox}[frametitle=Relax Time?]
		Kneading and shaping the dough makes it tight, and very elastic. As such, it is difficult to open the dough. Whenever you need to open a dough and it springs back you need to stop, cover the dough, and let it rest for $15$ minutes. After that, you will see that it opens much easier. Don't force the dough to open, or it will become a hard dense.
	\end{kaobox}
}

Put quite a lot of oil on a baking tray, and gently open the dough on the tray\sidenote{The dough, your hands, the tray and your soul will be covered in oil.}.
%
Press with your finger tips gently to help opening and shaping the dough on the tray, but not too much otherwise it will stick.

Cover it and let it rise again for $1h$.

Finally, right before putting it in the oven, mix salt, water and oil in a pot until it emulsify.
%
Dump your hands on the mixture generously and make the typical focaccia holes with your fingers. Here you want a lot of the mixture over the dough, which will also accumulate in the holes.

Put stuff on top, onions, rosemary, cherry tomatoes, whatever you want. Try to place them within the wholes\sidenote{Do not let the stuff too much higher than the dough, otherwise it will burn.}. 
%
Then brush the mixture on top once again, and slide into the oven.

\section{Baking}

Preheat the oven to $180-200^oC$. If it is possible, turn off the heat from bellow, here you want a gentle heat from above.
% 
Focaccia in the centre of the oven.
%
Bake until they are delicious-golden on top.
%
Brush oil on top right after it comes out from the oven.

\section{Experiments}

I need to measure better amounts for the mixture of salt, water and oil.
%
Try to substitute the yeast and biga with sourdough starter, usually $\frac{1}{3}$ of the flour mass, but be aware this will completely change the fermentation times. The relaxation times should remain the same.



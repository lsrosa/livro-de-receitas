\setchapterstyle{kao}
\setchapterpreamble[u]{\margintoc}
\chapter{Pizza}

\section{Ingredients}
The quantities are listed for a one-person pizza. Multiply them by the number of people you are feeding.

\begin{multicols}{2}
\begin{description}
	\item[120 g] Manitoba (wheat min.13g)
	\item[80 g] Water
	\item[12 g] Sourdough
	\item[2.4 g] Salt
	\item[4.8 g] Sugar
	\item[4.8 g] Oil \gls{evo}
\end{description}
\end{multicols}

\section{Preparation}
Mix all the ingredients, cover, and leave it to rest for $30$ minutes before kneading\sidenote{Read Chapter \ref{burger buns} for the why.}.
%
Relax and clean the space in the meanwhile.
%
After kneading, shape it into a tight ball and leave it to rest for $30$ minutes.

Divide the dough in equal parts\sidenote{I recommend using a scale because small differences in the balls sizes make big differences in the pizza sizes.} and shape them into tight, very tight balls\sidenote{Use a light dust of flour to help. The easiest way to tight is to use the scraper.}.
%
Place the balls in a container leaving a space about twice as big as the balls between them, and cover the contained tightly\sidenote{It is really important to cover the contained tightly so the dough does not dry while proofing, since sourdough has long proofing times. Covering with a towel is not enough!}.

Let the dough proof until they are fatty, touching each other so much that they are almost becoming one. That is when is time to open them and make the pizza.

To open the dough, use the scraper to cut the doughs on their edged and to pick them up from the tray. Throw them in a pile of flour and be careful to not deflate them.
%
Press from the centre to the edges gently, leaving the edges untouched.
%
Then, pulling it, you can use gravity, of any fancy technique. The secret here is not to deflate the dough. DO NOT USE A ROLLING PIN!

Shake the excess flour and put the pizza in a surface which makes it easy to slide into the oven. Make sure the pizza can dance on that surface\sidenote{Give it a horizontal shake to check is the pizza slides, otherwise dust a bit'o'flour on the bottom.}.
%
Put whatever you want as toppings.

\marginnote[0cm]{
	\begin{kaobox}[frametitle=Hot stone oven?]
		The pizza is originally cooked in stone ovens higher than $400^oC$.
		%
		That is, the dough will cook from below on the ho stone, and the edges from above. The high-temperature makes the pizza to cook super fast, leaving less time for the topping to dry.
		%
		As such, heat you oven as much as possible, and try to make a nice surface to cook the pizza from below, otherwise the liquids from the toppings will make the bottom of the pizza soggy/uncooked, and nobody likes a soggy bottom.
	\end{kaobox}
}

\section{Baking}

First, add some baking stones (or baking trays turned upside down) in the oven. You will slide the pizze in that surface.
%
Preheat the oven to the maximum for $1h$ before opening the dough and making the pizza.

Before 
Slide the pizza and wait a couple of minutes to bake.

%\section{Experiments}

%There is a bunch of technique



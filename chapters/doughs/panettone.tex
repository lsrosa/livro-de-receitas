\setchapterstyle{kao}
\setchapterpreamble[u]{\margintoc}
\chapter{Panettone}
First of all, it does not matter what they say, panettone is a bloody bread, and not a cake!
%
The second point is that panettone is like an ultimate brioche, you can think of it as the result of two challenges: "How much fat can I put in a dough?" and "How light can a dough get?" 

\section{Pre-Dough Ingredients}

\begin{multicols}{2}
\begin{description}
	\item[180 g] Strong Wheat Flour (manitoba)
	\item[90 g] Water
	\item[39 g] Sourdough
	\item[40 g] Room Temp. Butter	
	\item[40 g]	 Sugar
	\item[2 g] Malt
	\item[30 g] Yolk
\end{description}
\end{multicols}

\section{Final Dough Ingredients}
\begin{multicols}{2}
	\begin{description}
		\item[\rev{90 g}] Strong Wheat Flour
		\item[30 g]	Sugar
		\item[10 g] Honey
		\item[25 g] Room Temp. Butter	
		\item[30 g] Yolk
	\end{description}
\end{multicols}

\section{Souls}
\begin{multicols}{2}
	\begin{description}
		\item[75 g] Citrics paste
		\item[1] Vanilla bean
		\item[90 g] Raisings
		\item[60 g] Dried Fruits
		\item[??] Whisky
	\end{description}
\end{multicols}


\section{Preparation}

This is a two steps dough. We first make the first dough, let it rise until triple in time, and then knead it again with the rest of ingredients.

The first dough is simple, mix all the ingredients except the fats (butter and yolk), leave it rest for $1h$ and knead. 
%
Mix the yolks and butter until soft and fluffy, add them and knead until incorporated.

Take care while kneading it because the dough will probably not release from the bow as usual.
%
The final dough is sticky and hard to work, but I hope you will manage to make it\sidenote{Wet hands help a lot here.} into ball and put it to rise in a covered bow until it grows $2\times$ or $3\times$ in size.

For the final dough, mix the first dough with all ingredients but the fat, and repeat the kneading and fat incorporation.
%
As result, you will have what I call "base Panettone dough".


\marginnote[-10.5cm]{
	\begin{kaobox}[frametitle=Mind your steps:]
		Adding pastes, spirits, fruits and stuff will alter the water content of the dough, which might need to be adjusted on the fly. But take cake, because if you add too much liquid and flour you can ruin the structure of the dough, which is very very unforgivable.
	\end{kaobox}
}


\marginnote[-6.5cm]{
	\begin{kaobox}[frametitle=Adding souls:]
		The citric pastes and vanilla are added and mixed similarly to the butter. However, the solid and big stuff is better to be added onto the dough and mixed with some folds make over $30$ minutes intervals.
	\end{kaobox}
}

Panetonne traditionally has also many flavours embedded, which vary according to the taste.
%
Dried fruits soaked on spirit drinks are awesome. Vanilla seems to make it extra-luxurious.

To make citric pastes, you can slice oranges, mandarins, or limes\sidenote{God! Remove some of the whites too not make it to bitter.}, add sugar (half of the fruits weight), cook for 1h till soft, and blend it.

Make the dough into a ball, let it rest for half hour, shape it again and put it into a baking form. Cover it and let it rise until double the size\sidenote{You might need to be creative. Maybe you can use a big pan to cover it.}.

Remove the cover and let it rest for $30$ minutes, score a cross and add a teaspoon of butter in the centre of the cross.

\section{Baking}

Oven not so hot, $\approx180^oC$, it takes a while to cook.
%
After baking it, remove and let it rest completely upside down\sidenote{You will need to be creative again}.
  
\section{Experiments}

I am pretty sure some quantities need corrections.


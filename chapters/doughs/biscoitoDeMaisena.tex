\setchapterstyle{kao}
\setchapterpreamble[u]{\margintoc}
\chapter{Biscoito de Maizena/Sequilhos Amanteigados}

\section{Ingredients}

\begin{multicols}{2}
\begin{description}
	\item[250 g] Room temp. butter
	\item[500 g] Corn Starch
	\item[1 can] Sweetened condensed milk.
\end{description}
\end{multicols}

\section{Preparation}
Mix the corn starch\sidenote{It is a good practice to save a bit of corn starch for eventual mistakes.} and the butter.

Now it is the tricky park, add the condensed milk, but not all at once!
%
It usually absorbs a little less than a can, so go slow.

\marginnote[-1.5cm]{
	\begin{kaobox}[frametitle=Mind your steps:]
		After half a can, add it little by little mixing well between each addition, as it takes some time for it to come to the final consistency. This is important, do not fuck it up. If you add too much, you can fix with some extra corn starch.
	\end{kaobox}
}

The ideal consistency is not stick, and you can mold it without breaking.

Shape it in balls of $5~g$. Yep, tiny little balls. Get your scale and have fun. I like to put a bit of candied orange in the balls (see Recipe \ref{doce de laranja}).

\section{Baking}
Straight to center of the oven, pre-heated at a low-medium temperature until they are firm and dry.

Pay attention that ideally they should not brown, neither in the top or bottom. It should take something like $10~min$ or $15~min$ to bake, but keep an eye in the first tray to adjust your oven setting and to get an estimate of the baking time.

\section{Storing}
Store them in an air-tight container, possibly in multiple ones, as they tend to absorb a lot of humidity from the air and stop being super dry.
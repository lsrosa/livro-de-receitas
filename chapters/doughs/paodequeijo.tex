\setchapterstyle{kao}
\setchapterpreamble[u]{\margintoc}
\chapter{Pão de Queijo}
\rev{This is still a work in progress, attempt at your own risk}

\section{Ingredients}

\begin{multicols}{2}
\begin{description}
	\item[500 g] Polvilho azedo
	\item[500 g] Polvilho doce
	\item[150 g] Oil \gls{evo}
	\item[600 g] Milk
	\item[18 g] Salt
	\item[5~8 ] Small to medium Eggs
	\item[500 g] Half cured milk cheese
\end{description}
\end{multicols}

\section{Preparation}
Put both polvilhos\sidenote{Polvilho is a start made of mandioca, a traditional root in Brazil. You can use only the sour (azedo), but the pão de queijo will have a bit drier skin.} and salt in a big bow.

Put the oil and milk in a pan and heat it until the moment it starts to rise. You gotta be fast to turn off the fire in this exact moment. 
%
Go mixing with your biggest and most resistant wooden spoon\sidenote{Do not use plastic or metal spoons, the first will melt and deform and the second will scratch the bows and pan.}.

Pour the hot wet on the dry ingredients immediately, and mix vigorously with the wooden spoon.
%
At this point you must be brave, and start mixing this bloody hot dough with your hand, until it becomes smooth and uniform.

Wait the dough come to ``milk for babies'' temp\sidenote{Break it into small pieces to speed up the cooling.} and add the eggs and start incorporating them with your hands until is uniform and very well mixed.
%
The total amount of eggs varies, and you declare when by feel.
%
As you work the dough, it should become cohesive and stop sticking to your hand\sidenote{Wash your hands and come back to work the dough to test if it is not sticky.}.

Add the half cured cheese, and the cured cheese both coarsely grated and work it until everything is uniformly incorporated.

Make $50 g$ balls, or resort to any trick to make uniform sized balls.

\marginnote[-1cm]{
	\begin{kaobox}[frametitle=Which cheese?]
		For a half cured cheese is a simple one, the trick here is to leave it outside for a few days so it can dry a bit. The ideal point is when is has a hard skin outside but it is sill tender on the inside.
		%
		For the cured cheese, use whatever \textit{``flower''} you want. Parmegianno reggiano, pecorino, etc. 
	\end{kaobox}
}

\section{Baking}

On a baking tray, nothing fancy required. Oven at medium temp, place the tray in the centre of the oven.
%
They will be done when golden yellow on the outside with orange spots on the surface.
%
\textbf{DO NOT} bake until they look Caucasian, it is a waste of potentially good pães de queijo, ppl who do that should be in jail.

\section{Experiments}

Instead of grating the cured cheese, chop some of it into small pieces, so when you eat you find those pieces of good cheese.

%There is a bunch of technique


